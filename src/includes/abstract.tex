\setlength{\parskip}{0pt}

\clearpage
\chapter*{Abstract}

\subsubsection{EN}
In the last century, the theory of Quantum Mechanics helped pave the way for the creation of the transistor; a 
technological advancement that helped to create the electronic computer. As the computer was used more and more
for different applications, physicists pondered the limits of the miniaturization of the computer. Feynman
set the foundational ideas for a new kind of a computer; a Quantum computer, a computer that entales the Quantum
Mechanical laws. Although the foundational work has existed for over half a century, Quantum computers are still not
a commodity due to various difficulties on making error resistant qubits thus making chip manufacturing a very
difficult hurdle to overcome and algorithms hard to implement and produce reliable data. In this work we are going
to explore, analyze and code Quantum algorithms using the Qiskit SDK, a software development kit from IBM, to
create Quantum algorithms as Quantum circuits that implement arithmetic and logic operations of a classical Arithmetic
and Logic unit on a classical computer.

\subsubsection{EL}
Τον περασμένο αιώνα, η θεωρία της Κβαντομηχανικής βοήθησε να ανοίξει ο δρόμος για τη δημιουργία του τρανζίστορ- ένα 
τεχνολογικό απόκτημα που βοήθησε στη δημιουργία του ηλεκτρονικού υπολογιστή. Καθώς οι υπολογιστές χρησιμοποιούνταν όλο και περισσότερο
για διάφορες εφαρμογές, οι φυσικοί προβληματίστηκαν για τα όρια της σμίκρυνσης των υπολογιστών. Ο Feynman
έθεσε τις θεμελιώδεις ιδέες για ένα νέο είδος υπολογιστή - έναν κβαντικομηχανικό υπολογιστή, δηλαδή έναν υπολογιστή που αξιοποιεί την 
κβαντομηχανικούς νόμους για την λειτουργία του. Παρόλο που τα θεμέλια προυπάρχουν για πάνω από μισό αιώνα, οι κβαντικοί υπολογιστές δεν έχουν ακόμη
αναπτυχθεί λόγω διαφόρων δυσκολιών στην κατασκευή qubits - κβαντικά bits - ανθεκτικά στον θόρυβο από το περιβάλλον, καθιστώντας έτσι την κατασκευή ολοκληρωμένων τσιπ 
και την σχεδίαση χρήσιμων αλγορίθμων που να παράγουν αξιόπιστα δεδομένα μια πολύ δύσκολη υπόθεση. Σε αυτή την εργασία θα να εξερευνήσουμε, να αναλύσουμε και 
θα συγγράψουμε κβαντικούς αλγορίθμους χρησιμοποιώντας το Qiskit SDK, ένα κιτ ανάπτυξης λογισμικού από την IBM, για να να δημιουργήσουμε κβαντικούς αλγορίθμους ως κβαντικά
κυκλώματα που υλοποιούν αριθμητικές και λογικές πράξεις μιας Αριθμητικής και Λογικής Μονάδας όπως αυτές που υπάρχουν στους κλασικούς υπολογιστές.