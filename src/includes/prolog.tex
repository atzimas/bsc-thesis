\IfLanguageName{greek}{
  \chapter*{Πρόλογος}
}{
  \chapter*{Prolog}
}

At the end of the 20th century as computers were evolving into bigger, faster (and more exponentially power-hungry) electrical machines,
physists wondered about the physical limitations of such a systems. Regarded as the first to ponder and play with this kind of a problem is
nobel laureate, teacher of Physics and avid bongo player, Richard P. Feynman. On his 1982 talk in MIT \cite{Feynman1982}, he argued about the
use of a computing system that simulated quantum physical system he described a system abided by the laws of Quantum Mechanics that performed
computations, a \textit{Quantum Mechanical Computer}. Later, in 1984, he published a more specific article \cite{Feynman1984} describing a
more complete computing system with its fundamental components. Unfortunately, quantum computers did not produce any fruitful
and interesting results for many years. This status was quickly changed by Shor's work \cite{Shor1997} on algorithms for prime factorization.
This ground-breaking work was so explosive because of its premise, this algorithm could factorize large prime numbers in polynomial time.
This meant that many cryptographic algorithms that generated keys by prime generation could be broken substantially faster with a quantum
computer rather than a classical computer. In much modern times, with the work of superconducting qubits using Josephson's junctions
\cite{JOSEPHSON1962251} and Quantum Computers like IBM's System One \cite{IBMQuantum2021} and Google's Quantum processors \cite{Google2014}
quantum computers seem to be closer to the consumers than ever.