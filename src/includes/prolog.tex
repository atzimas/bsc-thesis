\IfLanguageName{greek}{
  \chapter*{Πρόλογος}
}{
  \chapter*{Prolog}
}

\textbf{Word count: Approximately 300-500 words, or around 3-5 paragraphs.}

The prologue section provides the reader with a broad overview of what the paper will cover, setting the stage for the detailed exploration to follow. Here, you might introduce your main theme or research question, provide some background on the topic, and/or briefly discuss the importance of the issue at hand. The tone is usually engaging, aiming to captivate the reader's interest right from the start.

\textbf{Here's an example of a prologue:}

"As the 21st century continues to unfold, the dynamics of international relations and diplomacy are changing at an unprecedented pace. Nations grapple with the interplay of historical conflicts and emerging challenges, continuously redefining their diplomatic strategies. This paper explores one such intricate terrain - the evolving nature of diplomacy in the context of [Your Specific Topic].

This journey began as a quest to understand how [Specific Issue] has shaped diplomatic interactions between nations, a topic that may initially appear esoteric, yet holds profound implications for the world's future. The motivation to delve into this topic was born out of observing [Relevant Personal Experience or Event].

The following chapters embark on a deep exploration of this subject, from its historical origins to its contemporary manifestations. Drawing on an array of sources and perspectives, this paper presents a comprehensive analysis of [Your Specific Topic], aiming to contribute to the ongoing discourse in this field. It is my hope that this work provokes thought, stimulates dialogue, and ultimately inspires further inquiry into the fascinating realm of diplomacy and international relations."
