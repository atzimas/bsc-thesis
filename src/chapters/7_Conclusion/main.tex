\chapter{Conclusions}

We have shown that it is possible to construct a very basic Quantum Arithmetic Logic Unit but due to the considerable compexity of the proposed designs of
the Quantum algorithms for each operation, the complete Unit is very susceptible to noise. We shall go over the details of each design.

The Adder-Subtractor Quantum circuit was based on the proposed circuit by Richard P. Feynman \cite{Feynman1984} but the design itself needs a $n$-qubit
output register to hold the computed output. There are other designs of Quantum adder circuits that use the \textit{Quantum Fourier Transform} or just 
\textit{QFT} to encode one of the inputs to the \textit{Fourier basis} transforming it from the initial state $p$ to $\phi{p}$ and then by using the qubits
from the other input perform conditional $R_z$ rotations on $\ket{p}$. This will compute $\ket{p+q}$ assuming the other input is $q$. Such a Quantum algorithm
and circuit was designed by Draper \cite{Draper2000} and it is used by Qiskit. Lastly, the Draper adder uses only $2n$ qubit instead of $3n+2$ qubits of this work's
Adder-Subtractor.

The Multiplier Quantum circuit is probably the number one contender for bringing unnecessary complexity to the complete system. This design uses $2n^2+2n$ qubits
which drives the complexity of the complete system by $n^2$. For a future work, a complete overhaul and change of the circuit is completely needed to drive down
the complexity of the circuit, which in turn can drive down the compute time needed to run on a Quantum compute, something that did happen and pushed us to only execute
parts of the complete design on real hardware instead of as a whole. A proposed alternative algorithm may be the Qiskit's own Multiplier Quantum circuit like the
Ruiz-Perez et al. \cite{Ruiz_Perez_2017} design (which is also called the RGQFT Multiplier). In contrast, the RGQFT Multiplier uses only $3n$ qubits in total.

Moreover, the NKO Comparator was a design implemented from previous work \cite{NKO2006}. The design is robust, concise and future work on the design may include adding
more calculations of other statuses like: detecting and overflow from addition/subtraction.

Lastly, for future work, the Quantum Arithmetic Logic Unit can be extended to include the operation of integer
division. This was not implemented in this thesis due to the inherent complexity of the integer division algorithm\cite{Thapliyal2021},
which requires a circuit that loops and compares values until a terminal state is reached.