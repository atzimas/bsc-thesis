\section{Signed Binary Numbers}

In general, negative numbers in decimal are notated by prefixing a minus
symbol ($-$) before the number. As we previously mentioned, computers
process information as sequences of bits. Thus negative numbers can not
be processed \say{as-is} by computers, thus making the need for an
encoding scheme.

One of the most simple encoding schemes to represent signed numbers
is called the \textit{signed-magnitude} representation. In this
scheme, the MSB (\textit{most-significant bit}) is considered to be the \textit{sign}; if that
bit is zero then the number is considered a positive integer
otherwise it's a negative integer. Binary numbers that do
not consider this scheme are called \textit{unsigned binary numbers}.

\begin{table}[ht]
    \centering
    \begin{tabular}{c|c}
        Signed Decimal notation & Signed Binary notation \\
        \hline
        $0$ & $0000$ \\
        $1$ & $0001$ \\
        $2$ & $0010$ \\
        $\vdots$ & $\vdots$ \\
        $6$  & $0110$ \\
        $7$  & $0111$ \\
        $-0$ & $1000$ \\
        $-1$ & $1001$ \\
        $\vdots$ & $\vdots$ \\
        $-6$ & $1110$ \\
        $-7$ & $1111$ \\
    \end{tabular}
    \caption{The table for 4-bit signed integers using the signed-magnitude representation}
\end{table}

Note that in this scheme there is a negative and positive zero. This is not ideal or economical 
and thus, this representation scheme is not used when designing computer circuits.

By using the \textit{signed-complement} representation scheme
signed integers are represented as their complement. There
are two kinds of signed-complement representations:

\begin{enumerate}
    \item one's complement and
    \item two's complement
\end{enumerate}

The one's complement of a negative number is the bitwise
inverse of a very bit of the corresponding positive counterpart
of that number. For example, the one's complement of the signed
decimal number $-12_{10}$ is obtained by:

\begin{enumerate}
    \item Take the unsigned bianry representation of the positive
    counterpart (12): $01100_2$
    \item Invert every bit (zero to one, one to zero): $10011_2$
\end{enumerate}
We should note that even with one's complement there are two
representations of zero; just like the signed-magnitude scheme.

The two's complement is simply obtained by taking the one's
complement of the negative number and adding one. For example,
the two's complement of $-12_{10}$ is obtained by:

\begin{enumerate}
    \item Take the unsigned bianry representation of the positive
    counterpart (12): $01100_2$
    \item Invert every bit (zero to one, one to zero): $10011_2$
    \item Add one: $10011 + 1 = 10100$
\end{enumerate}

We can see in the Table (2.5) the one's and two's complement of each
integer decimal that can be represented using four bits of information.

\begin{table}[ht]
    \centering
    \begin{tabular}{c|c|c}
        Signed Decimal notation & One's complement & Two's complement \\
        \hline
        $0$      & $0000$   & $0000$ \\
        $1$      & $0001$   & $0001$ \\
        $2$      & $0010$   & $0010$ \\
        $\vdots$ & $\vdots$ & $\vdots$ \\
        $6$      & $0110$   & $0110$ \\
        $7$      & $0111$   & $0111$ \\
        $-8$     & -        & $1000$ \\
        $-7$     & $1000$   & $1001$ \\
        $\vdots$ & $\vdots$ & $\vdots$ \\
        $-1$     & $1110$   & $1111$ \\
        $-0$     & $1111$   & - \\
    \end{tabular}
    \caption{The table for 4-bit signed integers using the signed-complement representation}
\end{table}

Two's complement is used universaly by computers becuase the representation simplifies the 
circuitry that implements arithmetic operations. Subtraction, for example, of two binary numbers $A-B$ can be
simplified as $A+B'+1$ where $B'$ is the one's complement of $B$.

For example, the difference of $1101_2 - 1011_2$ can be computed as follows:

\begin{enumerate}
    \item Take the two's complement of the second term: $1011_2\rightarrow0100_2 + 1 = 0101_2$
    \item Then compute $1101_2 + 0101_2 = 1\_0010_2$, excluding the overflow bit we get the indented
    difference
\end{enumerate}