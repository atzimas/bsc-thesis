\section{Boolean Algebra}

Boolean algebra is a branch of algebra that operates with binary variables - variable that can be assigned
either \textit{true} or \textit{false}. It defines three basic operations: negation ($\lnot$), conjunction
($\land$) and disjunction ($\lor$).

\begin{equation}
    A \land B
\end{equation}

\begin{equation}
    A \lor B
\end{equation}

\begin{equation}
    \lnot A
\end{equation}

The rules of operations for each operator is described in the following \textit{truth tables}. A truth
table is a table that displays all the possible binary values of a \textit{boolean variable}
and all the possible outputs. These operators are most commonly mapped to logic functions:
AND (conjunction), OR (disjunction) and NOT (negation), which can take either two input variables, for
the first two, or one input variable, for the last one. These function names are also used to map
the operation in the circuit model (see section 2.5).

The AND logic function states that the output will be \textit{true} (or logical 1) if and only if
both input variables are in the logical state 1.
\begin{table}[ht]
    \centering
    \begin{tabular}{cc|c}
        $a$ & $b$ & $a \land b = a \cdot b$ \\
        \hline
        $0$ & $0$ & $0$ \\
        $0$ & $1$ & $0$ \\
        $1$ & $0$ & $0$ \\
        $1$ & $1$ & $1$ \\
    \end{tabular}
    \caption{The truth table for the conjunction operator}
\end{table}

The OR logic function states that the output will be \textit{true} if both or at the very least
one of the input variables are in the logical state 1.
\begin{table}[ht]
    \centering
    \begin{tabular}{cc|c}
        $a$ & $b$ & $a \lor b = a + b$ \\
        \hline
        $0$ & $0$ & $0$ \\
        $0$ & $1$ & $1$ \\
        $1$ & $0$ & $1$ \\
        $1$ & $1$ & $1$ \\
    \end{tabular}
    \caption{The truth table for the disjunction operator}
\end{table}

The NOT logic function inverts the state of the input variable - if the input is in the logical state 0
then it is inverted to the logical state 1 and vice versa.
\begin{table}[ht]
    \centering
    \begin{tabular}{c|c}
        $a$ & $\lnot a = a'$ \\
        \hline
        $0$ & $1$ \\
        $1$ & $0$ \\
    \end{tabular}
    \caption{The truth table for the negation operator}
\end{table}

These are also called \textit{universal operators} because they can be used with each other to create
other logic functions like: the NAND (Not-AND), NOR (Not-OR), XOR (Exclusive-OR), XNOR (Exclusive-Not-OR)
are some of the most common functions other than the universal ones.

The NAND and NOR functions are self explanatory, they are just the negated AND and OR functions.
\begin{table}[ht]
    \centering
    \begin{tabular}{cc|c}
        $a$ & $b$ & $\lnot(a \land b) = (a \cdot b)'$ \\
        \hline
        $0$ & $0$ & $1$ \\
        $0$ & $1$ & $1$ \\
        $1$ & $0$ & $1$ \\
        $1$ & $1$ & $0$ \\
    \end{tabular}
    \caption{The truth table for the NAND function}
\end{table}
\hspace{1cm}
\begin{table}[ht]
    \centering
    \begin{tabular}{cc|c}
        $a$ & $b$ & $\lnot(a \lor b) = (a + b)'$ \\
        \hline
        $0$ & $0$ & $1$ \\
        $0$ & $1$ & $0$ \\
        $1$ & $0$ & $0$ \\
        $1$ & $1$ & $0$ \\
    \end{tabular}
    \caption{The truth table for the NOR function}
\end{table}

The XOR function is the most elaborate logic function. The output is only true when only one of the 
inputs are true. This function is sometimes symbolised by the $\oplus$ operator.

\begin{table}[ht]
    \centering
    \begin{tabular}{cc|c}
        $a$ & $b$ & $(\lnot a \land b) \lor (a \land \lnot b) = a'\cdot b + a\cdot b' = a \oplus b$ \\
        \hline
        $0$ & $0$ & $1$ \\
        $0$ & $1$ & $1$ \\
        $1$ & $0$ & $1$ \\
        $1$ & $1$ & $0$ \\
    \end{tabular}
    \caption{The truth table for the NAND function}
\end{table}