\section{The Digital Logic Gates}

Computers are constructed using \textit{digital components} - components
that fundamentaly operate with binary inputs. These components manipulate
voltages to produce a specific output - a logical low voltage is considered
the zero or false state and a logical high voltage is considered the one or
truth state. These components are called \textit{logic gates} or \textit{logic components}.

Just like Boolean Algebra, these components adhere to the same rules. The universal
operations - the AND, OR and NOT functions - using the \textit{circuit model} of
computation are treated as circuits that accomplish the same operation.

The AND gate is a digital gate that takes two 1-bit inputs and outputs their logical
conjunction. Thus the truth table and operator are analogous to its Boolean counterpart.

\begin{figure}[ht]
    \centering
    \begin{circuitikz}
        \draw (0,0) node[and port] (p) {};
        \draw (p.in 1) node[left] {$a$};
        \draw (p.in 2) node[left] {$b$};
        \draw (p.out) node[right] {$a \land b = a \cdot b$};
    \end{circuitikz}
    \caption{The circuit symbol for the AND function}
\end{figure}

The OR gate is a digital gate that takes two 1-bit inputs and outputs their logical
disjunction. Again the truth table and operator are analogous to its Boolean counterpart.

\begin{figure}[ht]
    \centering
    \begin{circuitikz}
        \draw (0,0) node[or port] (p) {};
        \draw (p.in 1) node[left] {$a$};
        \draw (p.in 2) node[left] {$b$};
        \draw (p.out) node[right] {$a \lor b = a + b$};
    \end{circuitikz}
    \caption{The circuit symbol for the OR function}
\end{figure}

Lastly, the NOT gate takes a 1-bit input and outputs its logical negation.

\begin{figure}[ht]
    \centering
    \begin{circuitikz}
        \draw (0,0) node[not port] (p) {};
        \draw (p.in) node[left] {$a$};
        \draw (p.out) node[right] {$\lnot a = a'$};
    \end{circuitikz}
    \caption{The circuit symbol for the AND function}
\end{figure}

We would like to point out that there are also digital logic gates that symbolise the
other compount logic functions - NAND, NOR, XOR and XNOR.

\begin{figure}[ht]
    \centering
    \begin{subfigure}{0.3\textwidth}
        \centering
        \begin{circuitikz}
            \draw (0,0) node[nand port] (p) {};
            \draw (p.in 1) node[left] {$a$};
            \draw (p.in 2) node[left] {$b$};
            \draw (p.out) node[right] {$\lnot(a \land b) = (a \cdot b)'$};
        \end{circuitikz}
        \caption{}
    \end{subfigure}
    \begin{subfigure}{0.3\textwidth}
        \centering
        \begin{circuitikz}
            \draw (0,0) node[nor port] (p) {};
            \draw (p.in 1) node[left] {$a$};
            \draw (p.in 2) node[left] {$b$};
            \draw (p.out) node[right] {$\lnot(a \lor b) = (a + b)'$};
        \end{circuitikz}
        \caption{}
    \end{subfigure}
    \begin{subfigure}{0.3\textwidth}
        \centering
        \begin{circuitikz}
            \draw (0,0) node[xor port] (p) {};
            \draw (p.in 1) node[left] {$a$};
            \draw (p.in 2) node[left] {$b$};
            \draw (p.out) node[right] {$a \oplus b$};
        \end{circuitikz}
        \caption{}
    \end{subfigure}
    \begin{subfigure}{0.3\textwidth}
        \centering
        \begin{circuitikz}
            \draw (0,0) node[xnor port] (p) {};
            \draw (p.in 1) node[left] {$a$};
            \draw (p.in 2) node[left] {$b$};
            \draw (p.out) node[right] {$(a \oplus b)' = a \odot b$};
        \end{circuitikz}
        \caption{}
    \end{subfigure}
    \caption{The digital logic gates for (a) NAND, (b) NOR, (c) XOR and (d) XNOR functions}
\end{figure}