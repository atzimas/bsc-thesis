\section{The Binary Numeral System}

The \textit{binary numeral system} is a positional numeral system that uses two values -
zero (0) and one (1) - to notate numbers. A number notated using two values is called
a \textit{binary number}.

Binary numbers are expressed as a sequence of digits $d_id_{i-1}\ldots d_1d_0$
where $i$ is the position of the digit $d$. A more concise expression of
a binary number is:

\begin{equation}
    A=\sum_{i=0}^{n}d_i\times2^i
\end{equation}

which expression is derived by the general definition

\begin{equation}
    A=\sum_{i=0}^{n}d_i\times b^i
\end{equation}

where $b$ is the \textit{base} or \textit{radix} of the numeral system the
number is expressed in and $n$ is the total digits of the number. The digits
of a binary number are notoriously called \textit{binary digits} or \textit{bits}
for short.

We note that the left-most bit of a binary number is called the \textit{most-significant bit}
and the right-most \textit{least-significant bit}, \textit{MSB} and \textit{LSB} resprectively.

To differentiate \textit{decimal numbers} from binary numbers we notate that
number with its base in subscript. So, the decimal number $110$ will be
notated as $110_{10}$ and the binary number $1010$ as $1010_2$.

We can covnert any binary number to its decimal numeral notation by
following the expression 2.2. So that $1010_2$ can be expressed in decimal as:

\begin{equation}
    1010_2 = 1 \times 2^3 + 0 \times 2^2 + 1 \times 2^1  + 0 \times 2^0 = 8 + 0 + 2 + 0 = 10_{10}
\end{equation}

We can also do the reverse, convert a decimal number into its binary notation, by recursively
dividing by the binary base and noting down the remainders until the operations gives a zero.
For example, $13_{10}$ is converted in binary notation as follows:

\begin{enumerate}
    \item $13 \div 2 = 6\text{ (remainder 1)}$
    \item $6 \div 2 = 3\text{ (remainder 0)}$
    \item $3 \div 2 = 1\text{ (remainder 1)}$
    \item $1 \div 2 = 0\text{ (remainder 1)}$
\end{enumerate}

If we read the remainders in reverse order we acquire the binary respresentation of the decimal number $13_{10}$ which
is $1101_2$.

The binary system is useful to all modern computers as they operate by using transistors
as saturated switches. This was first realised by Shannon \cite{Shannon1937} who
noted that any switching circuit can obey the rules of \textit{binary logic}. This kind
of logic was developed by Boole \cite{Boole1847} and later concretely founded by
Huntington \cite{Huntington1904}. This logic is now know as \textit{Boolean algebra}
and we will talk in more depth in section 2.4.