\section{Logic Circuits}

We can use the logic gates introduced in section 2.5 to create \textit{logic circuits}
- circuits that compute some Boolean complex function. For example, we can produce
a logic circuit that computes the sum of two bits.

As we have noted in section 2.2 the rules of addition are:

\begin{enumerate}
    \item $0 + x = x$
    \item $1 + 1 = 0\text{ carrying }1$
\end{enumerate}

Note that the sum $S$ can be produced by the XOR gate and the carry $C$ with the AND gate. Putting those
two components together we get the following circuit:

\begin{figure}[ht]
    \centering
    \begin{circuitikz}
        \draw (0, 4)node[xor port] (xor){}
        (0, 2)node[and port] (and){}
        (xor.in 1) node[left=0.5cm](a) {$a$}
        (xor.in 2) node[left=0.5cm](b) {$b$}
        
        (a.east) to[short,-*] (xor.in 1) |- (and.in 1)
        (b.east) to[short,-*] ($(b.east)!.5!(xor.in 2)$) coordinate (branch)
            -- (xor.in 2)
        (branch) |- (and.in 2);
        \draw (xor.out) node[right] {$S$};
        \draw (and.out) node[right] {$C$};
    \end{circuitikz}
    \caption{The logic circuit of the half adder}
\end{figure}

\begin{table}[ht]
    \centering
    \begin{tabular}{cc|cc}
        $A$ & $B$ & $S$ & $C_{out}$ \\
        \hline
        $0$ & $0$ & $0$ & $0$ \\
        $0$ & $1$ & $1$ & $0$ \\
        $1$ & $0$ & $1$ & $0$ \\
        $1$ & $1$ & $0$ & $1$ \\
    \end{tabular}
    \caption{The truth table of the half adder}
\end{table}

This digtal logic circuit is called \textit{half adder}.
Why is it called the \textit{half} adder? This is because this circuit is not aware of any previous carry, thus
it is not a \textit{complete} circuit. As we have seen from section 2.2 carrying is very important to the calculation
of the sum.

The circuit that does take into account for a previous carry bit is called the \textit{full adder}.
The previous carry is called the \textit{carry-in} bit and it's the third input for this circuit.

\begin{figure}[ht]
    \centering
    \includegraphics[scale=.7]{images/5_Implementation/classical_fulladder_diagram.pdf}
    \caption{The logic circuit of the full adder using two half adders}
\end{figure}

\begin{table}[!ht]
    \centering
    \begin{tabular}{ccc|cc}
        $a$ & $b$ & $C_{in}$ & $S$ & $C_{out}$ \\
        \hline
        $0$ & $0$ & $0$ & $0$ & $0$ \\
        $0$ & $1$ & $0$ & $1$ & $0$ \\
        $1$ & $0$ & $0$ & $1$ & $0$ \\
        $1$ & $1$ & $0$ & $0$ & $1$ \\
        $0$ & $0$ & $1$ & $1$ & $0$ \\
        $0$ & $1$ & $1$ & $0$ & $1$ \\
        $1$ & $0$ & $1$ & $0$ & $1$ \\
        $1$ & $1$ & $1$ & $1$ & $1$ \\
    \end{tabular}
    \caption{The truth table of the full adder}
\end{table}

\newpage

We can see by the diagrams that the full adder circuit can be made out of
two half adder circuits. These kind of circuits, that are made from other
simple circuits, are called \textit{combinational circuits}.

By placing full adders in sequence we can make another combinational circuit
called the \textit{ripple carry adder}. This circuit can compute the sum
of $n$-bit inputs. It's called a \textit{ripple} carry adder because in order
to compute the $n$-th sum bit the circuit must first compute the $n-1$-th
sum bit thus some latency is present on the circuit. We must note at this point
that there are other better implementations of $n$-bit adders but it is out of
the scope of this work. In figure (2.7) we can see an example of a 4-bit
ripple carry adder which computes, for each input bit $A_i$, $B_i$ and $C_{in_i}$,
the sum bit $S_i$ and the carry-out bit and passes it to the next full-adder block as
the carry-in bit $C_{in_{i+1}}$. The logic continues until the circuit computes the sum bits for
all the input bits.

\begin{figure}[ht]
    \centering
    \includegraphics[width=14cm]{images/2_Classical_Computing/ripple_carry_adder.pdf}
    \caption{The logic circuit of a 4-bit ripple carry adder}
\end{figure}