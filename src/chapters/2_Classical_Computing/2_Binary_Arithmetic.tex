\section{Binary Arithmetic}

Just like in the decimal system, the binary system, defines operations that operate
using binary numbers.

Addition in binary is as simple as addition in decimal. The symbol that carries the
binary addition is \say{$+$} and the rules or \textit{axioms} are as followed:

\begin{enumerate}
    \item $0 + x = x\text{ , where $x$ is a bit}$. This rule defines zero as the
    identity element of addition, which means that any binary number that is
    added with zero will equal itself.
    \item $1 + 1 = 0\text{ , carrying 1}$. This is called carry
    and it is similar to when we add the decimal numbers together and they
    exceed the base (10). This produces a one that will be added to next
    sum of terms.
\end{enumerate}

For example, we shall add $1101_2 + 1011_2$:

\begin{table}[ht]
    \centering
    \begin{tabular}{c cccc}
        $1$ & $1$   & $1$  & $1$  & \\
        $ $ & $1$  & $1$ & $0$ & $1$ \\
        $+$ & $1$ & $0$ & $1$ & $1$ \\
        \hline
        $1$ & $1$ & $0$ & $0$ & $0$ \\
    \end{tabular}
    \caption{An example of binary addition}
\end{table}

Note the top row of digits, which are the carry bits from each column addition.

Binary Subtraction is also simple.

\begin{enumerate}
    \item $0 - 0 = 0$
    \item $0 - 1 = 1\text{ , borrowing }1$. This is similar to the carry except
    the next two digits that are going to be subtracted need to subtract the borrow too.
    \item $1 - 0 = 1$
    \item $1 - 1 = 0$
\end{enumerate}

For example, we shall subtract $1101_2 - 1011_2$:

\begin{table}[ht]
    \centering
    \begin{tabular}{c cccc}
        $ $ & $ $ & $1$ & $ $ & \\
        $ $ & $1$ & $1$ & $0$ & $1$ \\
        $-$ & $1$ & $0$ & $1$ & $1$ \\
        \hline
        $ $ & $0$ & $0$ & $1$ & $0$ \\
    \end{tabular}
    \caption{An example of binary subtraction}
\end{table}

Lastly, we will discuss the axioms of binary multiplication. The product of any two binary
numbers can be produced by firstly calculating the \textit{partial product} of these two
numbers. The partial product is calculated by the following axioms:

\begin{enumerate}
    \item $0 \times x = 0$
    \item $1 \times x = x$. One (1) is the identity element of binary multiplication.
\end{enumerate}

The product can then be calculated by summing all the partial products.

For example, we shall multiply $1101_2 \times 1011_2$:

\begin{table}[ht]
    \centering
    \begin{tabular}{c cccccccc}
        $ $      & $ $ & $ $ & $ $ & $ $ & $1$ & $1$ & $0$ & $1$ \\
        $\times$ & $ $ & $ $ & $ $ & $ $ & $1$ & $0$ & $1$ & $1$ \\
        \hline
        $ $ & $ $ & $ $ & $ $ & $ $ & $1$ & $1$ & $0$ & $1$ \\
        $ $ & $ $ & $ $ & $ $ & $1$ & $1$ & $0$ & $1$ & $ $ \\
        $ $ & $ $ & $ $ & $0$ & $0$ & $0$ & $0$ & $ $ & $ $ \\
        $+$ & $ $ & $1$ & $1$ & $0$ & $1$ & $ $ & $ $ & $ $ \\
        \hline
        $ $ & $1$ & $0$ & $0$ & $0$ & $1$ & $1$ & $1$ & $1$ \\
    \end{tabular}
    \caption{An example of binary multiplication}
\end{table}