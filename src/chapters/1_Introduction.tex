\chapter{Εισαγωγή}

Σε αυτή την εργασία θα εξερευνήσουμε κάποια λογικά κυκλωματικά στοιχεία που μπορούν να χρησιμοποιηθούν για την κατασκευή μιας απλής Αριθμητικής
Λογικής Μονάδας. Αρχικά θα γίνει μια ανάλυση της αρχικής ιδέας του περίφημου θεωρητικού φυσικού και δασκάλου, Richard P. Feynman
για την θεωρητική προσέγγιση ενός υπολογιστή που θα χρησιμοποιεί του κβαντικούς νόμους της φυσής για να τελεί έργο των ανθρώπων.

Έπειτα, θα μιλήσουμε σε έκταση για μερικά βασικά θεμέλια της κλασικής όσο και της κβαντικής θεωρίας της Υπολογιστικής που θα χρειαστούν για την
κατανόηση της εργασίας. Στο δεύτερο κεφάλαιο θα γίνει μια βιβλιογραφική ανασκόπηση του πεδίου της Κβαντικής Υπολογιστικής και Λογικής Σχεδίασης
με σκοπό ο αναγνώστης να αποκτήσει ένα ολοκληρωμένο πλαίσιο για την βαθύτερη κατανόηση της εργασίας.

Μετά την βιβλιογραφική ανασκόπηση ακολουθούν τα κυριώς κεφάλαια της εργασίας. Τα κύρια κεφάλαια έχουν χωριστεί σε δυο μερη: το πρώτο μέρος αναλύει
την υλοποίηση των κβαντικών λογικών κυκλωμάτων χρησιμοποιώντας την γλώσσα προγραμματισμού Python και του Qiskit, ένα Σετ Ανάπτυξης Λογισμικού
(Software Development Kit)

\section{"Κβαντομηχανικοί Υπολογιστές"}

\section{Στοιχεία της Κλασικής Υπολογιστικής}

\section{Στοιχεία της Κβαντικής Υπολογιστικής}
