\section{Landauer's Principle}

When one bit of information is erased, the least amount of energy is expelled to the enviroment given by the
following equation:

\begin{equation}
    E = k_BT\ln2
\end{equation}

where $k_B$ is the \textit{Boltzmann constant} and $T$ is the temperature of the enviroment. This is
know as the \textit{Landauer's Principle}\cite{Landauer1961}. We can use this equation to calculate the total work produced
by the computer by defining a Boolean function $f:\{0,1\}^m\to\{0,1\}^n$ that takes $m$-input bits and
produces $n$-output bits - this funtion will represent all of the logical operations of the computer.
If we consider that $m>>n$, this means that some bits are erased each time the computer carries a computation\cite{Marmorkos2024},
then the total energy loss using equation (3.8) is:

\begin{equation}
    E_{total}=(m-n)k_BT\ln2
\end{equation}

Because Equation (3.9) gives the minimum amount of energy produced, the real amount of energy produced by
modern computers is much higher that $E_{total}$.