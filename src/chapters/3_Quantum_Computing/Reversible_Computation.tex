\section{Reversible Computation}

\say{Classical} computers use three fundamental logic gates to process binary information:
the AND, OR and NOT gates as we have introduced them in the previous chapter. A very simple
question can rise from analyzing these logic gate \say{Can we know the state of the inputs by
just looking at the outputs}? In other words, \say{Can we reverse the computation done by those
logic gates}?

The NOT gate is \textit{reversible} because its output is the negation of the input and we can infer
the input by just negating again. This is also a postulate of the Boolean Algebra:

\begin{equation}
    \lnot(\lnot x) = x
\end{equation}

The AND and OR gates are not reversible. For the AND gate, we have three outputs
that result in a zero (0) and only one possibility for an output of one (1), thus we cannot infer
for three outputs what the inputs are with absolute certainty. For the OR gate is the same
but instead of three outputs of zero this gate has three possibilities of outputs of one (1).
It seems like we lost some information of what the inputs were. In fact we do lose information
in form of heat \cite{Landauer1961}. We can analyze for the other compound gates: NAND, NOR, XOR
and XNOR. We can find out easily that all of those gate are not reversible for the same reasons.

The most important thing to consider is the energy loss in the form of heat. It has been shown \cite{Bennett1982}
that by making logic components reversible we can save a lot of power generated by the computer.
These kind of components are called \textit{reversible logic gates}. This idea of reversible computation
is the basis that Feynman used to impose the idea of the Quantum Computer, a computer that uses
reversible components that obey the laws of Quantum Mechanics \cite{Feynman1982}.