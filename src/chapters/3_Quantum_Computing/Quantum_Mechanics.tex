\section{Quantum Mechanics for Quantum Computing}

Before we begin to operate a quantum computer or make quantum circuits we have to discuss
what are the underline laws that govern those systems.

Quantum Mechanics is a mathematical framework that is used to create physical theories or
to describe properties of physical systems. It is also a framework that helps physicists
to describe nature at the microscopic level, something that Classical Mechanics failed to do.

This mathematical framework has four \textit{axioms} that are used to interconnect the physical
with the mathematical world.

\subsection{The Axioms of Quantum Mechanics}
\subsubsection{Axiom One: The State of a Quantum System}

The state of a quantum system can be represented as a complex vector, which vector is in the
complex Hilber space. The state of a quantum system is also called the \textit{state vector} of
that quantum system. In quantum computing usually systems are \textit{two level systems}, that are systems
that have two states (binary systems).

The \textit{qubit} is the quantum equivalent of the classical bit - instead of storing classical information
(zeros or ones) is stores quantum information expressed using some \textit{basis states}. The most frequent
state base is the \textit{computational basis}. The computational basis describes two linearly seperable
and orthogonal unitary vectors $\{\ket{0},\ket{1}\}\in\mathbb{H}^2$. With those two states we can encode
information and process it using quantum gates. Physicaly these two state have to represent
a physical state of the system. Classical computers use two voltage levels (+5V and +0V, for example), in 
quantum computers we can use many physical properties of a quantum system like: the spin of an
electron (up or down), the polarization of a photon (horizontal or vertical) or even the energy levels of an atom.

A general state $\ket{\phi}$ of a system can be written as a superposition of the basis states $\ket{0}$ and $\ket{1}$:

\begin{equation}
    \ket{\phi} = a\ket{0} + b\ket{1} = \begin{bmatrix}
        a\\
        b
    \end{bmatrix}
\end{equation}

where $a,b\in\mathbb{C}$ are the complex coefficients called the \textit{probability amplitudes} of
the state vector $\ket{\phi}$. We can also notate them using the name of the state vector

\begin{equation}
    \ket{\phi} = \phi_0\ket{0} + \phi_1\ket{1} = \begin{bmatrix}
        \phi_0\\
        \phi_1
    \end{bmatrix}
\end{equation}

\subsubsection{Axiom Two: Time Evolution of a Quantum System}

The time evolution of a quantum system $\ket{\phi(t)}$ can be expresed by unitary operators
of the form $U(t, t_0)$ as follows:

\begin{equation}
    \ket{\phi(t)} = U(t, t_0)\ket{\phi(t_0)}
\end{equation}

Also if an operator is hermitian ($A=A^\dag$) then the operator $U=e^{iA}$ is unitary because:

\begin{equation}
    UU^\dag=e^{iA}(e^{iA})^\dag=e^{iA}e^{-iA^\dag}=e^{iA}e^{-iA}=I
\end{equation}

\subsubsection{Axiom Three: Measurement}

By now we can encode information and we can define its evolution in time, but it would be
gratuitous if we could not measure the outcome of any computation. Thus the third axiom states
that to make a measurement on a quantum system $\ket{\phi}\in\mathbb{H}$, defined using the
basis state $\ket{i}\in\mathbb{H}$, then a measurement can occur on $\ket{\phi}=\sum_{i}^{n}a_i\ket{i}$.
The outcome is one of the states $\ket{i}$ of the system with a probability to be in that state $|a_i|^2$.
After the outcome, the system \textit{collapses} to the that state $\ket{i}$.

\subsubsection{Axiom Four: Composite Systems}

We can combine $n$ independent quantum systems that are expressed as state vectors $\ket{\phi_i}\in\mathbb{H}_i$ where
$i=0,1,\hdots,n$ as a \textit{composite quantum system} $\ket{\phi}$. This can be done by applying the tensor product
to each of the independent systems

\begin{equation}
    \ket{\phi} = \ket{\phi_0}\otimes\ket{\phi_1}\otimes\hdots\otimes\ket{\phi_n}=\ket{\phi_0\phi_1\hdots\phi_n}
\end{equation}

That state vector $\ket{\phi}$ is a complex vector in the complex Hilber space

\begin{equation}
    \mathbb{H}=\mathbb{H}_0\otimes\mathbb{H}_1\otimes\hdots\mathbb{H}_n
\end{equation}

As we mentioned previously, in Quantum Computing, we study two-level systems - systems with two basis states.
This means that the composite system $\ket{\phi}$ of $n$ independent two-level quantum systems
is in a complex Hilbert space of $2^n$ dimensions.