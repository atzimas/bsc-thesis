\section{The Dirac Notation}

We already are familiar with the standard vector notation. Let $\vec{v}$ be a $n$-dimensional vector
whom's elements are real numbers. This is notated as:

\begin{equation}
    \vec{v}=\sum_{i=1}^{n}v_i\hat{d_i}\in\mathbb{R}^n
\end{equation}

where $\hat{d_i}$ is the unit vector of the $i$-th dimension. For a three-dimensional
real space the vector $\vec{v}$ would be notated as:

\begin{equation}
    \vec{v}=v_1\hat{d_1}+v_2\hat{d_2}+v_3\hat{d_3}\in\mathbb{R}^3
\end{equation}

The notation is similar for elements in the complex space $\mathbb{C}^n$. The only real difference
is the nomenclature where vectors that are in that space are called \textit{complex vectors}.

We can also notate vectors as a $n\times1$ column matrix, where each element of that matrix
is the vector's corresponding element:

\begin{equation}
    \vec{v}=\begin{bmatrix}
        v_1\\
        v_2\\
        \vdots\\
        v_n
    \end{bmatrix}\in\mathbb{R}^n
\end{equation}

In Quantum Mechanics, vectors are represented using a different kind of notation, the \textit{Dirac notation}
or \textit{bra-ket notation}. This notation was introduced by the American physicist and electrical engineer Paul
Dirac in 1939 \cite{Dirac1939}. This introduced two new symbols: the \textit{bra} (symbolised with a $\bra{}$)
which represents a vector quantity whose elements are a vertical $1\times n$ matrix, and the \textit{ket}
(symbolised with a $\ket{}$) which represents a vector quantity whose elements are a horizontal $n\times1$ matrix.
We also note that in Quantum Mechanics all vectors (and their elements) are in a complex space. For example,
let $\vec{v}$ be a vector in the complex space $\mathbb{V}^n$, we would notate it in Dirac notation as:

\begin{equation}
    \ket{v}=\begin{bmatrix}
        v_1\\
        v_2\\
        \vdots\\
        v_n
    \end{bmatrix}
    \text{, where }v_i\in\mathbb{C},\ket{v}\in\mathbb{V}^n
\end{equation}
This is also read as \say{ket $v$}. Note that the arrow symbol ($\rightarrow$) on top of label-name of the vector
$\vec{v}$ is absent with this notation.

Subsequently, the \say{bra $v$} is the \textit{conjugate transpose} or the \textit{Hermitian conjugate} (symbolised with
$\dag$ and pronounced as \say{dagger}) of the $\ket{v}$

\begin{equation}
    \bra{v}=\begin{bmatrix}
        v_1^*&v_2^*&\hdots&v_n^*
    \end{bmatrix}=
    \begin{bmatrix}
        v_1\\
        v_2\\
        \vdots
        v_n
    \end{bmatrix}^\dag
\end{equation}

\subsection{Inner Product}

The inner product of two complex vectors $\ket{a},\ket{b}\in\mathbb{V}^n$ is defined as

\begin{equation}
    \braket{a|b}=\begin{bmatrix}
        a_1^*&a_2^*\hdots a_n^*
    \end{bmatrix}\times
    \begin{bmatrix}
        b_1\\
        b_2\\
        \vdots\\
        b_n
    \end{bmatrix}=
    \sum_{i=1}^{n}a_i^*b_i=c\text{, where }a_i,b_i,c\in\mathbb{C}
\end{equation}

There are also three properties defined for the inner product:

\begin{equation}
    \braket{a|b}=\braket{b|a}^*
\end{equation}

\begin{equation}
    \braket{a|\alpha b+\beta c}=\alpha\braket{a|b}+\beta\braket{a|c}\text{, where }\alpha,\beta\in\mathbb{C}
\end{equation}

\begin{equation}
    \braket{\alpha a+\beta c|b}=\alpha^*\braket{a|b}+\beta^*\braket{c|a}\text{, where }\alpha,\beta\in\mathbb{C}
\end{equation}

The inner product of two vectors $\ket{a},\ket{b}\in\mathbb{V}^n$ is zero when the vectors are orthogonal to each other.

\subsection{Hilbert Space}

When in an abstract complex vector space $\mathbb{V}^n$ the operation of the inner product is defined then
that space is also called a \textit{Hilbert space}. In Quantum Computing, every vector that represents the
state of the quantum system is a vector in Hilbert space.